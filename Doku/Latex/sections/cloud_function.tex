\section{Backend-Services mit Cloud Run Functions}

\subsection{Ziel der Backend-Funktionalität}

Zur Umsetzung der serverseitigen Logik wurde Google Cloud Run in Verbindung
mit Cloud Functions eingesetzt. Ziel ist es, dem Frontend eine klar definierte
REST-Schnittstelle bereitzustellen, über die sicher auf Anwendungsdaten,
Datenbanken und Cloud-Storage-Ressourcen zugegriffen werden kann.

Direkte Zugriffe aus dem Frontend auf Datenbank oder Storage werden dabei
bewusst vermieden. Sämtliche Datenmanipulationen erfolgen ausschließlich
über die bereitgestellten Backend-Services.

\subsection{Aufruf von Cloud Run in der Google Cloud Console}

Die Erstellung und Verwaltung der Backend-Services erfolgt über die
Google Cloud Console. Nach Auswahl des Projekts wird im Navigationsmenü
der Dienst \textit{Cloud Run} aufgerufen. Innerhalb dieses Bereichs werden
alle laufenden Services zentral angezeigt und verwaltet.

\begin{figure}[H]
    \centering
    \includegraphics[width=0.9\textwidth]{Images/Function1.png}
    \caption{Navigation zu Cloud Run innerhalb der Google Cloud Console}
\end{figure}

\subsection{Übersicht der bestehenden Services}

In der Serviceübersicht werden alle bereitgestellten Cloud-Run-Services
angezeigt. Jeder Service stellt einen eigenständigen HTTP-Endpunkt dar
und kann unabhängig von anderen Services skaliert, aktualisiert oder
neu bereitgestellt werden.

In diesem Projekt werden mehrere Services eingesetzt, darunter:
\begin{itemize}
    \item API-Services zur Verarbeitung von Anwendungsdaten,
    \item Funktionen zum Upload von Dateien,
    \item Funktionen zum Abruf von Inhalten aus dem Cloud Storage.
\end{itemize}

\begin{figure}[H]
    \centering
    \includegraphics[width=0.9\textwidth]{Images/Function2.png}
    \caption{Übersicht der bestehenden Cloud-Run-Services}
\end{figure}

\subsection{Erstellung einer neuen Cloud-Run-Function}

Zur Erstellung eines neuen Backend-Services wurde die Option
\textit{Funktion schreiben} gewählt. Dabei handelt es sich um eine
Cloud-Run-Function, die direkt über einen Inline-Editor erstellt und
bereitgestellt werden kann.

Während der Erstellung werden grundlegende Parameter definiert, darunter:
\begin{itemize}
    \item der eindeutige Name des Services,
    \item die Region (z.\,B. europe-west3),
    \item die verwendete Laufzeitumgebung (Node.js),
    \item die Authentifizierungsart,
    \item das Abrechnungsmodell.
\end{itemize}

\begin{figure}[H]
    \centering
    \includegraphics[width=0.9\textwidth]{Images/Function3.png}
    \caption{Konfiguration eines neuen Cloud-Run-Services}
\end{figure}

\subsection{Authentifizierung, Skalierung und Abrechnung}

Für den erstellten Service wurde ein öffentlicher HTTP-Endpunkt bereitgestellt,
sodass das Frontend direkt mit der Cloud-Run-Function kommunizieren kann.
Die Zugriffskontrolle kann optional über IAM oder Identity Tokens erweitert
werden.

Die Skalierung erfolgt automatisch auf Basis der eingehenden Anfragen.
Die minimale Anzahl an Instanzen wurde auf null gesetzt, wodurch im
Leerlauf keine Kosten entstehen.

Die Abrechnung erfolgt anfragebasiert, sodass Kosten nur bei tatsächlicher
Nutzung des Services anfallen.

\begin{figure}[H]
    \centering
    \includegraphics[width=0.9\textwidth]{Images/Function4.png}
    \caption{Abrechnungs- und Skalierungseinstellungen des Cloud-Run-Services}
\end{figure}

\subsection{Einbindung in die Gesamtarchitektur}

Die Cloud-Run-Functions bilden die zentrale Schnittstelle zwischen
Frontend, Datenbank und Cloud Storage. Das Frontend kommuniziert ausschließlich
über diese Services mit dem Backend.

Durch diese Architektur wird sichergestellt, dass:
\begin{itemize}
    \item keine direkten Datenbankzugriffe aus dem Client erfolgen,
    \item sensible Operationen serverseitig kontrolliert werden,
    \item die Anwendung horizontal skalierbar bleibt,
    \item sicherheitsrelevante Logik zentral im Backend gebündelt ist.
\end{itemize}
