\section{Firebase Authentication}

\subsection{Zugriff auf den Authentication-Bereich}

Die Konfiguration der Benutzerverwaltung erfolgt über die Firebase Console.
Nach Auswahl des Projekts \textit{Doku} wird im linken Navigationsmenü der
Bereich \textit{Authentication} ausgewählt. Dieser Bereich dient als zentrale
Verwaltungsoberfläche für alle Aspekte der Authentifizierung.

\begin{figure}[H]
    \centering
    \includegraphics[width=0.9\textwidth]{Images/Auth1.png}
    \caption{Aufruf des Authentication-Bereichs in der Firebase Console}
\end{figure}

\subsection{Übersicht der Nutzerverwaltung}

Nach dem Aufruf des Authentication-Moduls wird standardmäßig der Reiter
\textit{Nutzer} angezeigt. In diesem Bereich verwaltet Firebase alle registrierten
Benutzerkonten der Anwendung.

Für jeden Nutzer speichert Firebase unter anderem:
\begin{itemize}
    \item den verwendeten Authentifizierungsanbieter,
    \item den Zeitpunkt der Erstellung des Kontos,
    \item den Zeitpunkt der letzten Anmeldung,
    \item eine eindeutige Benutzer-ID (UID).
\end{itemize}

Die UID dient als stabiler Identifikator und wird für die Zuordnung von
benutzerspezifischen Daten sowie für Autorisierungsentscheidungen in
nachgelagerten Services verwendet.

\begin{figure}[H]
    \centering
    \includegraphics[width=0.9\textwidth]{Images/Auth2.png}
    \caption{Nutzerübersicht innerhalb von Firebase Authentication}
\end{figure}

\subsection{Konfiguration der Anmeldemethoden}

Die verfügbaren Anmeldemethoden werden im Reiter \textit{Anmeldemethode}
konfiguriert. In diesem Projekt wurden bewusst mehrere Authentifizierungsanbieter
aktiviert, um unterschiedliche Anmeldewege zu ermöglichen.

Konkret wurden folgende Anbieter eingerichtet:
\begin{itemize}
    \item \textbf{E-Mail-Adresse / Passwort} als klassische Form der Anmeldung,
    \item \textbf{Google} als externer OAuth-2.0-Anbieter.
\end{itemize}

Die Aktivierung erfolgt über die Schaltfläche \textit{Neuen Anbieter hinzufügen}.
Jeder Anbieter kann unabhängig voneinander aktiviert oder deaktiviert werden.

\begin{figure}[H]
    \centering
    \includegraphics[width=0.9\textwidth]{Images/Auth3.png}
    \caption{Auswahl und Aktivierung der Anmeldemethoden}
\end{figure}

\subsection{Registrierung der Web-Anwendung}

Um Firebase Authentication im Frontend nutzen zu können, wurde eine Web-Anwendung
im Firebase-Projekt registriert. Dieser Schritt ist notwendig, damit das Frontend
eindeutig dem Projekt zugeordnet werden kann.

Nach der Registrierung stellt Firebase eine projektspezifische Konfiguration zur
Verfügung, die unter anderem folgende Parameter enthält:
\begin{itemize}
    \item \texttt{apiKey}
    \item \texttt{authDomain}
    \item \texttt{projectId}
    \item \texttt{storageBucket}
    \item \texttt{appId}
\end{itemize}

Diese Konfigurationsdaten werden im Frontend verwendet, um das Firebase SDK zu
initialisieren und eine Verbindung zu den Firebase-Diensten herzustellen.

\begin{figure}[H]
    \centering
    \includegraphics[width=0.9\textwidth]{Images/Auth4.png}
    \caption{Firebase SDK Konfiguration für die Web-Anwendung}
\end{figure}

\subsection{Einbindung des Firebase SDK}

Für die Einbindung von Firebase im Frontend wurde das modulare JavaScript SDK
verwendet. Die Installation erfolgt über den Node Package Manager (npm).
Nach der Initialisierung des SDK kann die Authentication-Funktionalität direkt
im Client genutzt werden.

Die Authentifizierung erfolgt vollständig über Firebase, wodurch keine
Passwörter oder sicherheitskritischen Zugangsdaten im eigenen Backend
gespeichert oder verarbeitet werden müssen.

\subsection{Zusammenfassung}

Firebase Authentication übernimmt in diesem Projekt die vollständige
Benutzerverwaltung und Authentifizierung. Durch die Nutzung eines externen
3rd-Party-Dienstes wird eine sichere, skalierbare und wartbare Lösung
bereitgestellt, die sich nahtlos in weitere Google-Cloud-Dienste integrieren lässt.
