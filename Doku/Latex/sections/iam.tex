\section{Identity and Access Management (IAM)}

\subsection{Ziel des IAM-Konzepts}

Zur Absicherung der Cloud-Infrastruktur wurde Google Cloud Identity and Access
Management (IAM) eingesetzt. Ziel ist es, Zugriffe auf Cloud-Ressourcen
granular zu steuern und sicherzustellen, dass jede Komponente der Anwendung
nur über die minimal notwendigen Berechtigungen verfügt.

IAM bildet damit die Grundlage für:
\begin{itemize}
    \item kontrollierten Zugriff auf Cloud-Run-Services,
    \item abgesicherte Datenbank- und Storage-Zugriffe,
    \item Trennung von Frontend, Backend und Infrastruktur,
    \item Nachvollziehbarkeit von Zugriffen über Audit-Logs.
\end{itemize}

\subsection{Aufruf des IAM-Bereichs}

Die Konfiguration der Zugriffsrechte erfolgt über den Bereich
\textit{IAM und Verwaltung} innerhalb der Google Cloud Console.
Nach Auswahl des Projekts wird der Menüpunkt \textit{IAM} geöffnet.

\begin{figure}[H]
    \centering
    \includegraphics[width=0.9\textwidth]{Images/IAM1.png}
    \caption{Navigation zum IAM-Bereich in der Google Cloud Console}
\end{figure}

\subsection{Vergabe von Zugriffsrechten}

Innerhalb der IAM-Übersicht können Hauptkonten, Gruppen und Dienstkonten
mit spezifischen Rollen ausgestattet werden. Die Vergabe neuer Berechtigungen
erfolgt über die Funktion \textit{Zugriffsrechte erteilen}.

Dabei wird zunächst ein Hauptkonto oder Dienstkonto ausgewählt und
anschließend eine oder mehrere Rollen zugewiesen, die definieren,
welche Aktionen auf Projektebene erlaubt sind.

\begin{figure}[H]
    \centering
    \includegraphics[width=0.9\textwidth]{Images/IAM2.png}
    \caption{Erteilen von Zugriffsrechten über IAM}
\end{figure}

\subsection{Service Accounts für Backend-Services}

Für die Backend-Services wurde ein dediziertes Dienstkonto
(\textit{api-service}) verwendet. Dieses Dienstkonto wird von den
Cloud-Run-Services genutzt, um kontrolliert auf weitere Cloud-Ressourcen
zugreifen zu können.

Dem Dienstkonto wurden gezielt Rollen zugewiesen, unter anderem:
\begin{itemize}
    \item Zugriff auf Cloud Storage,
    \item Zugriff auf Cloud SQL,
    \item Schreibrechte für Logs,
    \item Zugriff auf Secrets im Secret Manager.
\end{itemize}

Durch diese Trennung wird vermieden, dass Cloud-Run-Services mit
übergeordneten oder globalen Rechten ausgeführt werden.

\begin{figure}[H]
    \centering
    \includegraphics[width=0.9\textwidth]{Images/IAM3.png}
    \caption{Zugewiesene Rollen für das Service Account des API-Services}
\end{figure}

\subsection{Einbindung des Service Accounts in Cloud Run}

Das konfigurierte Dienstkonto wird direkt dem Cloud-Run-Service
zugewiesen. Dies erfolgt bei der Bereitstellung oder Aktualisierung
eines Services im Reiter \textit{Sicherheit}.

Durch diese Zuweisung werden alle Zugriffe des Cloud-Run-Services
automatisch im Kontext des gewählten Dienstkontos ausgeführt.

\begin{figure}[H]
    \centering
    \includegraphics[width=0.9\textwidth]{Images/IAM4.png}
    \caption{Zuweisung des Service Accounts zu einem Cloud-Run-Service}
\end{figure}

\subsection{Revisionen und sichere Aktualisierung}

Jede Änderung an der Konfiguration eines Cloud-Run-Services führt
zur Erstellung einer neuen Revision. Diese Revision enthält neben
dem Container-Image auch die Sicherheits- und IAM-Einstellungen.

Durch dieses Revisionsmodell bleibt jederzeit nachvollziehbar,
welche Konfiguration aktiv ist und welche Berechtigungen verwendet werden.

\begin{figure}[H]
    \centering
    \includegraphics[width=0.9\textwidth]{Images/IAM5.png}
    \caption{Bereitstellung einer neuen Revision mit aktualisierten Sicherheitsparametern}
\end{figure}

\subsection{Sicherheitsbewertung}

Durch den Einsatz von IAM in Kombination mit dedizierten Service Accounts
wird das Prinzip der minimalen Rechtevergabe (Least Privilege) umgesetzt.
Frontend-Komponenten besitzen keinerlei direkte Zugriffsrechte auf
Cloud-Ressourcen.

Alle sicherheitskritischen Operationen werden ausschließlich durch
Backend-Services durchgeführt, die klar definierte Rollen und Berechtigungen
besitzen.
